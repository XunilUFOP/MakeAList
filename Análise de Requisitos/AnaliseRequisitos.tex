% !TeX spellcheck = pt_BR
\documentclass[12pt,oneside,a4paper,article]{abntex2}

% ----------------------------------------------------------
% PACOTES
% ----------------------------------------------------------

% ---
% Pacotes fundamentais
% ---
%\usepackage{multicol}           
\usepackage{cmap}               % Mapear caracteres especiais no PDF
\usepackage[T1]{fontenc}        % seleção de códigos de fonte.
\usepackage[utf8]{inputenc}     % determina a codificação utiizada (conversão automática dos acentos)
\usepackage{makeidx}            % cria o indice
\usepackage{hyperref}           % controla a formação do índice
\usepackage{lastpage}           % usado por abntex2-fichacatalografica.tex
\usepackage{indentfirst}        % Identa o primeiro parágrafo de cada seção.
%\usepackage{nomencl}           % Lista de simbolos
\usepackage{graphicx}           % Pacote para inserir imagens JPEG/JPG e PNG
\usepackage{xcolor}             % Pacote de cores
\usepackage{graphicx,color}     % Pacote para definir cores diferentes para fonte
\usepackage{textcomp}

\usepackage{mdframed}
\usepackage{minted}
\usemintedstyle{monokai}

\newcolumntype{C}[1]{>{\centering\let\newline\\\arraybackslash\hspace{0pt}}m{#1}}

% ---
% Pacotes de citações
% ---
\usepackage[brazilian,hyperpageref]{backref}     % Paginas com as citações na bibl
\usepackage[alf]{abntex2cite}   % Citações padrão ABNT

% ---
% Alterações para melhor aparência do modelo article
% ---
% Elimina a marcação de capítulo nas seções e permite não se utilizar \chapter{}
\renewcommand{\thesection}{\arabic{section}}
% Altera o tamanho da fonte para Referências, com o tamanho de seção, já que não são usados capítulos.
\renewcommand{\ABNTEXchapterfontsize}{\Large}
% ---


% Title Page
\title{Make.a.List\\Análise de Requisitos}

\author{João Pedro Santos de Moura\\
		Marcus Vinicius Nunes Calisto\\
		Oto Braz Assunção}

\local{Departamento de Computação e Sistemas\\
	   Instituto de Ciências Exatas e Aplicadas\\
	   Universidade Federal de Ouro Preto}
	
\date{20 de novembro de 2015}


\begin{document}
\maketitle

\pagebreak

\tableofcontents

\pagebreak

\listoffigures

\pagebreak

\listoftables

\part{Introdução}
\section{Uma Breve Introdução}
	É comum que professores lecionem mais de um disciplina durante os períodos acadêmicos.
	Durante o período acadêmico, como parte da distribuição de créditos, é comum que os docentes reservem uma parte dos mesmos para resolução de listas de exercícios.

	Dependendo da disciplina, o processo de confecção e resolução destas listas pode se tornar custoso em relação ao tempo, pois a ementa da disciplina pode ser extensa e isso pode levar o professor a gastar muito tempo selecionando questões de acordo com a disciplina e seus tópicos.

\part{Descrição do Problema}
	\section{Descrevendo o Problema}
		\subsection{O Problema dos Professores}
			Tendo em vista que professores tendem a ministrar mais de uma turma, muitas vezes de diferentes disciplinas, o processo criativo de listas de exercícios pode se tornar extenuante nesses casos.
		
		\subsection{O Problema dos Alunos}
			Ao receber uma lista, é comum o aluno resolvê-la e esperar uma possível correção por parte do professor, seja durante uma aula específica ou agendando um horário.
			O problema é o tempo que isso custa ao aluno, que poderia ter acesso as respostas logo após fazer o exercício ou a partir de uma determinada semana definida pelo
			professor. Assim, o aluno seria capaz de conferir suas respostas sem tomar também o tempo do professor e o horário da aula seria usado para sanar as dúvidas
			mais relevantes.
			
\part{Objetivo}
	\section{O Objetivo}
		O objetivo geral é facilitar o processo de criação e correção de listas de exercício tanto para alunos quanto para professores.
		\subsection{Visando os Professores}
			O objetivo da aplicação é agilizar o processo de confecção de listas de exercícios por parte dos professores, além de poupar tempo com atendimento aos alunos
			que buscam respostas de tais questões presentes nessas listas.

		\subsection{Visando os Alunos}
			Evitar que o aluno perca muito tempo para consultar o professor com relação à respostas de exercícios e seja capaz de conferir por si próprio se sua resolução de
			uma determinada questão está certa ou não.
			
\part{Escopo da Aplicação}
	\section{O Escopo}
		O ambiente acadêmico será o escopo da aplicação mas nada impede que uma empresa possa adaptar o sistema de forma a gerar essas listas como  testes em cursos de
		treinamentos mas para esta aplicação em específico, o espoco é o ambiente acadêmico, que é onde mais se usa esse tipo ferramenta.

\part{Descrição do Produto}
	\section{Descrevendo o Produto}
		
		\subsection{A Web}
			A web é um amplo ambiente de serviços e propicio a diversas novas ideias, tendo isso em mente, uma ferramenta, seja qual for sua função, é interessante neste ambiente e pode ser de grande ajuda a diversas pessoas. Um fato importante é o acesso liberado a todos, e apenas a web tem condições de suportar tamanha diversidade de serviços e de oferecer os alicerces necessários para a exposição de novas ideia. Acreditando nessas bases, o produto desenvolvido pela Xunil Xolusions XpreX, tem total capacidade de ser utilizado com as opções oferecidas pela WEB.
		
		\subsection{A aplicação}
			Usar o contexto para mostrar porque é importante que ela seja desenvolvida nessa plataforma, o seu objetivo, funcionamento, para quem e porquê eles devem usar.
\part{Caso de Uso}
	\section{Casos de Uso}
		Nesta seção são apresentados os artefatos correspondentes aos casos de uso do produto, bem como os seus respectivos diagramas.
		
		\subsection{Caso de Uso I - Cadastrar Usuário}
		
		\textbf{Escopo:} make.a.list 
		
		\textbf{Nível:} ??
		
		\textbf{Atores primários:} Admin, Aluno, Professor
		
		\textbf{Interessados:} Admin: Efetuação de cadastro a fim de acessar o sistema. 
		
		Aluno: Efetuação de cadastro a fim de acessar o sistema.
		
		Professor: Efetuação de cadastro a fim de acessar o sistema.
		
		\textbf{Pré-condições:} Acesso à Internet.
		
		\textbf{Garantias de sucesso:} Usuário devidamente adicionado ao banco de dados. Detalhes e link de confirmação enviados para o \textit{e-mail}.  
		
		\textbf{Cenário de sucesso principal:}
		
		\begin{enumerate}
			\item Usuário acessa a página web;
			
			\item Usuário seleciona o link \textit{Cadastrar};
			
			\item Usuário preenche devidamente os campos do formulário de cadastro.
			
			\item Usuário aceita os Termos de Uso do sistema
			
			\item Usuário confirma o cadastro
			
			\item Sistema confirma que o cadastro foi efetuado com sucesso.
		\end{enumerate}
		
		\textbf{Extensões:} Passos 3 à 5 \textrightarrow \ Campos preenchidos erroneamente:
		
			\begin{enumerate}[label=\alph*.]	
				\item Processo de cadastro é interrompido
				\item Usuário recebe \textit{feedback} relativos ao problema	
			\end{enumerate}
		
		\textbf{Frequência de ocorrência:} Usado moderadamente
			
		\subsection{Caso de Uso II - Editar Usuário}
		
		\textbf{Escopo:} make.a.list 
		
		\textbf{Nível:} ??
		
		\textbf{Atores primários:} Admin, Aluno, Professor
		
		\textbf{Interessados:} Admin: Realizar alterações em campos editáveis após cadastro.
		
		Aluno: Realizar alterações em campos editáveis após cadastro.
		
		Professor: Realizar alterações em campos editáveis após cadastro.
		
		\textbf{Pré-condições:} Acesso à Internet. Usuário autenticado.
		
		\textbf{Garantias de sucesso:} Campo devidamente alterado no banco de dados. Atualização do valor do campo exibido no sistema. 
		
		\textbf{Cenário de sucesso principal:}
		
		\begin{enumerate}
			\item Usuário acessa a página web;
			
			\item Usuário seleciona o link \textit{Fazer Login};
			
			\item Usuário realiza o processo de autenticação;
			
			\item Usuário acessa seu Perfil;
			
			\item Usuário seleciona modifica o campo desejado;
			
			\item Usuário confirma a edição;
			
			\item Perfil do usuário é atualizado;
			
		\end{enumerate}
		
		\textbf{Extensões:} Passo 3 \textrightarrow \ Usuário ou senha inválidos:
		
		\begin{enumerate}[label=\alph*.]	
			\item Processo de login é interrompido;
			\item Erro é notificado ao usuário;
			\item Dados são solicitados novamente;	
			
		Passos 5-6 \textrightarrow \ Campos preenchidos erroneamente:
		
		\begin{enumerate}[label=\alph*.]	
			\item Processo de edição é interrompido.
			\item Usuário recebe \textit{feedback} relativos ao problema	
		\end{enumerate}
		
		\end{enumerate}
		
		\textbf{Frequência de ocorrência:} Usado frequentemente.
		
		\subsection{Caso de Uso III - Remover Usuário}
				
		\textbf{Escopo:} make.a.list 
		
		\textbf{Nível:} Admin
		
		\textbf{Atores primários:} Admin
		
		\textbf{Interessados:} Admin: Realizar remoção de um usuário cadastrado.
		
		\textbf{Pré-condições:} Acesso à Internet. Privilégio de Administrador.
		
		\textbf{Garantias de sucesso:} Usuário devidamente removido do banco de dados. Confirmação exibida.
		
		\textbf{Cenário de sucesso principal:}
		
		\begin{enumerate}
			\item Admin acessa a página web;
			
			\item Admin realiza o processo de autenticação;
			
			\item Admin pesquisa usuário à ser removido;
			
			\item Admin aperta o botão remover;
			
			\item Admin confirma que ele realmente deseja remover o usuário;
			
			\item Usuário é removido do banco de dados;
			
			\item Mensagem de confirmação de remoção é exibida.
			
		\end{enumerate}
		
		\textbf{Extensões:} Passo 2 \textrightarrow \ Usuário ou senha inválidos:
		
		\begin{enumerate}[label=\alph*.]	
			\item Processo de login é interrompido;
			\item Erro é notificado ao usuário;
			\item Dados são solicitados novamente;	
			
		\end{enumerate}
		
		\textbf{Frequência de ocorrência:} Usado raramente.
		
		\subsection{Caso de Uso IV - Logar}
		
		\textbf{Escopo:} make.a.list 
		
		\textbf{Nível:} ??
		
		\textbf{Atores primários:} Admin, Aluno, Professor
		
		\textbf{Interessados:} Admin: Efetuação de autenticação a fim de acessar o sistema. 
		
		Aluno: Efetuação de autenticação a fim de acessar o sistema.
		
		Professor: Efetuação de autenticação a fim de acessar o sistema.
		
		\textbf{Pré-condições:} Acesso à Internet. Usuário previamente cadastrado.
		
		\textbf{Garantias de sucesso:} Redirecionamento para página inicial do usuário.
		
		\textbf{Cenário de sucesso principal:}
		
		\begin{enumerate}
			\item Usuário acessa a página web;
			
			\item Usuário seleciona o link \textit{Logar};
			
			\item Usuário informa campos \textit{usuário} e \textit{senha};
			
			\item Usuário pressiona botão \textit{Logar};
			
			\item Usuário redirecionado para sua página inicial;
			
		\end{enumerate}
		
		\textbf{Extensões:} Passos 3-4 \textrightarrow \ Usuário ou senha inválidos:
		
		\begin{enumerate}[label=\alph*.]	
			\item Processo de login é interrompido;
			\item Erro é notificado ao usuário;
			\item Dados são solicitados novamente;
		\end{enumerate}
		
		\textbf{Frequência de ocorrência:} Usado frequentemente.
		
		\subsection{Caso de Uso V - Adicionar Disciplina}
		
		\textbf{Escopo:} make.a.list 
		
		\textbf{Nível:} ??
		
		\textbf{Atores primários:} Admin
		
		\textbf{Interessados:} Admin: Criação de uma nova disciplina no sistema.
		
		Aluno: Gerar listas de exercícios da disciplina criada.
		
		Professor: Criar listas de exercícios da disciplina criada.
		
		\textbf{Pré-condições:} Acesso à Internet. Privilégios de Administrador
		
		\textbf{Garantias de sucesso:} Disciplina é criada e adicionada ao banco de dados. Mensagem de confirmação é exibida.
		
		\textbf{Cenário de sucesso principal:}
		
		\begin{enumerate}
			\item Admin realiza processo de login no sistema;
			
			\item Admin seleciona o link \textit{Adicionar disciplina};
			
			\item Admin preenche o formulário de criação da disciplina;
			
			\item Criação da disciplina é confirmada;
			
			
		\end{enumerate}
		
		\textbf{Extensões:} Passo 1 \textrightarrow \ Usuário ou senha inválidos:
		
		\begin{enumerate}[label=\alph*.]	
			\item Processo de login é interrompido;
			\item Erro é notificado ao usuário;
			\item Dados são solicitados novamente;
		\end{enumerate}
		
		Passo 3 \textrightarrow \ Campos preenchidos erroneamente:
		
		\begin{enumerate}[label=\alph*.]	
			\item Processo de criação é interrompido.
			\item Admin recebe \textit{feedback} relativos ao problema.	
		\end{enumerate}
		
		
		\textbf{Frequência de ocorrência:} Usado moderadamente.
	
		\subsection{Caso de Uso VI - Adicionar \textit{tópico} de exercício à disciplina}
		
		\textbf{Escopo:} make.a.list 
		
		\textbf{Nível:} ??
		
		\textbf{Atores primários:} Admin, Professor
		
		\textbf{Interessados:} Admin: Adição de um tópico à determinada disciplina.
		
		Professor: Adição de um tópico à determinada disciplina.
		
		\textbf{Pré-condições:} Acesso à Internet. Admin ou professor autenticados.
		
		\textbf{Garantias de sucesso:} Novo tópico é atribuído à disciplina. Mensagem de confirmação é exibida.
		
		\textbf{Cenário de sucesso principal:} 
		
		\begin{enumerate}
			\item Usuário acessa a página web;
			
			\item Usuário realiza processo autenticação
			
			\item Usuário seleciona a disciplina.
			
			\item Usuário seleciona link de adição de novo tópico.
			
			\item Usuário preenche formulário de adição de nova \textit{tópico}.
			
			\item Criação do \textit{tópico} é confirmada.
			
		\end{enumerate}
		
		\textbf{Extensões:} 
		
		Passo 1 \textrightarrow \ Usuário ou senha inválidos:
		
		\begin{enumerate}[label=\alph*.]	
			\item Processo de autenticação é interrompido;
			\item Erro é notificado ao usuário;
			\item Dados são solicitados novamente;
		\end{enumerate}
		
		Passo 2 \textrightarrow \ Campos preenchidos erroneamente:
		
		\begin{enumerate}[label=\alph*.]	
			\item Processo de adição do tópico é interrompido
			\item Usuário recebe \textit{feedback} relativos ao problema	
		\end{enumerate}
		
		\textbf{Frequência de ocorrência:} Usado moderadamente
	
		\subsection{Caso de Uso VII - Adicionar Questão}
		
		\textbf{Escopo:} make.a.list 
		
		\textbf{Nível:} ??
		
		\textbf{Atores primários:} Admin, Professor
		
		\textbf{Interessados:} Admin: Adição de uma questão ao banco de dados de questões.
		
		Professor: Adição de uma questão ao banco de dados de questões.
		
		\textbf{Pré-condições:} Acesso à Internet. Admin ou professor autenticados.
		
		\textbf{Garantias de sucesso:} Nova questão é adicionada ao banco de dados de questões.
		
		\textbf{Cenário de sucesso principal:} 
		
		\begin{enumerate}
			\item Usuário acessa a página web;
			
			\item Usuário realiza processo autenticação;
			
			\item Usuário seleciona o link \textit{Adicionar Questão};
			
			\item Usuário preenche o formulário de adição questão.	
			
			\item Criação de questão é confirmada.
			
		\end{enumerate}
		
		\textbf{Extensões:} 
		
		Passo 1 \textrightarrow \ Usuário ou senha inválidos:
		
		\begin{enumerate}[label=\alph*.]	
			\item Processo de autenticação é interrompido;
			\item Erro é notificado ao usuário;
			\item Dados são solicitados novamente;
		\end{enumerate}
		
		Passo 2 \textrightarrow \ Campos preenchidos erroneamente:
		
		\begin{enumerate}[label=\alph*.]	
			\item Processo de adiçao de questão é interrompido.
			\item Usuário recebe \textit{feedback} relativos ao problema.	
		\end{enumerate}
		
		\textbf{Frequência de ocorrência:} Usado moderadamente
	
		\subsection{Caso de Uso VIII - Gerar Lista - Professor}
		
			\textbf{Escopo:} make.a.list 
			
			\textbf{Nível:} ??
			
			\textbf{Atores primários:} Admin, Professor
			
			\textbf{Interessados:} Admin: Gerar uma lista de exercícios com sucesso.
			
			Professor: Gerar uma lista de exercícios contendo as respostas dos mesmos.
			
			\textbf{Pré-condições:} Acesso à Internet. Admin ou Professor autenticados.
			
			\textbf{Garantias de sucesso:} Nova aba é aberta e a lista em formato \textit{PDF} é exibida. Em seguida, o usuário pode fazer o download da lista.
			
			\textbf{Cenário de sucesso principal:} 
			
			\begin{enumerate}
				\item Usuário acessa a página web;
				
				\item Usuário realiza processo autenticação;
				
				\item Usuário seleciona o link \textit{Gerar Lista};
				
				\item Usuário escolha a disciplina dentre uma lista;
				
				\item Usuário escolhe quantidade dos exercícios e os tópicos que devem ser abordados pelos mesmos.	
				
				\item Nova aba no navegador é aberta exibindo a lista gerada.
				
			\end{enumerate}
			
			\textbf{Extensões:} 
			
			Passo 1 \textrightarrow \ Usuário ou senha inválidos:
			
			\begin{enumerate}[label=\alph*.]	
				\item Processo de autenticação é interrompido;
				\item Erro é notificado ao usuário;
				\item Dados são solicitados novamente;
			\end{enumerate}
			
			Passo 5 \textrightarrow \ Número de exercícios informado é maior que a quantidade disponível:
			
			\begin{enumerate}[label=\alph*.]	
				\item Notificação do problema é exibido.
				\item Alerta perguntando se o usuário deseja uma lista com todos exercícios possíveis.
			\end{enumerate}
			
			\textbf{Frequência de ocorrência:} Usado frequentemente
			
	
		\subsection{Caso de Uso IX - Gerar lista - Aluno}
		
		
		\textbf{Escopo:} make.a.list 
		
		\textbf{Nível:} ??
		
		\textbf{Atores primários:} Admin, Aluno
		
		\textbf{Interessados:} Admin: Gerar uma lista de exercícios referentes aos tópicos informados.
		
		Professor: Gerar uma lista de exercícios referentes aos tópicos informados.
		
		\textbf{Pré-condições:} Acesso à Internet. Admin ou Aluno autenticados.
		
		\textbf{Garantias de sucesso:} Nova aba é aberta e a lista em formato \textit{PDF} é exibida. Em seguida, o usuário pode fazer o download da lista.
		
		\textbf{Cenário de sucesso principal:} 
		
		\begin{enumerate}
			\item Usuário acessa a página web;
			
			\item Usuário realiza processo autenticação;
			
			\item Usuário seleciona o link \textit{Gerar Lista};
			
			\item Usuário escolha a disciplina dentre uma lista;
			
			\item Usuário escolhe quantidade dos exercícios e os tópicos que devem ser abordados pelos mesmos.	
			
			\item Nova aba no navegador é aberta exibindo a lista gerada.
			
		\end{enumerate}
		
		\textbf{Extensões:} 
		
		Passo 1 \textrightarrow \ Usuário ou senha inválidos:
		
		\begin{enumerate}[label=\alph*.]	
			\item Processo de autenticação é interrompido;
			\item Erro é notificado ao usuário;
			\item Dados são solicitados novamente;
		\end{enumerate}
		
		Passo 5 \textrightarrow \ Número de exercícios informado é maior que a quantidade disponível:
		
		\begin{enumerate}[label=\alph*.]	
			\item Notificação do problema é exibido.
			\item Alerta perguntando se o usuário deseja uma lista com todos exercícios possíveis.
		\end{enumerate}
		
		\textbf{Frequência de ocorrência:} Usado frequentemente
	
		\subsection{Caso de Uso X - Recuperar uma Lista Gerada}
		
		\textbf{Escopo:} make.a.list 
		
		\textbf{Nível:} ??
		
		\textbf{Atores primários:} Admin, Aluno, Professor
		
		\textbf{Interessados:} Admin: Recuperar lista através de código único.
		
		Aluno: Recuperar lista através de código único.
		
		Professor: Recuperar lista através de código único.
		
		\textbf{Pré-condições:} Acesso à Internet. Usuário autenticado.
		
		\textbf{Garantias de sucesso:} Nova aba é aberta mostrando a lista recuperada.
		
		\textbf{Cenário de sucesso principal:} 
		
		\begin{enumerate}
			\item Usuário acessa a página web;
			
			\item Usuário realiza processo autenticação;
			
			\item Usuário seleciona o link \textit{Recuperar Lista};
			
			\item Usuário informa código da lista.
			
			\item Lista é recuperada e aberta em nova aba do navegador.			
		\end{enumerate}
		
		\textbf{Extensões:} 
		
		Passo 1 \textrightarrow \ Usuário ou senha inválidos:
		
		\begin{enumerate}[label=\alph*.]	
			\item Processo de autenticação é interrompido;
			\item Erro é notificado ao usuário;
			\item Dados são solicitados novamente;
		\end{enumerate}
		
		Passo 5 \textrightarrow \ Código inexistente:
		
		\begin{enumerate}[label=\alph*.]	
			\item Notificação da inexistência da lista é exibida para o usuário.
		\end{enumerate}
		
		\textbf{Frequência de ocorrência:} Usado frequentemente
		

\part{Classes}
	\section{Classes}
		Nesta seção são apresentadas as classes que integram a aplicação, bem como um diagrama para cada e um diagrama de classes onde mostra
		as relações entre elas.
		
		\subsection{Usuário}
			Classe genérica para compartilhar atributos em comum entre as subclasses Administrador, Professor e Aluno.
	
				\subsubsection{Administrador}
					Usuário capaz de inserir, remover e atualizar outros usuários Professores e Alunos bem como adicionar, editar e remover todas as categorias do
					sistema. Tem todos os privilégios do sistema.
				\subsubsection{Professor}
				Usuário capaz de inserir e remover disciplinas, tópicos e questões.
	
				\subsubsection{Aluno}
					Usuário com o menor nível de privilégio, capaz de recuperar a lista de acordo com um código único e visualizar as respostas depois de uma data definida
					por quem gerou a lista.

		\subsection{Disciplina}
			A classe Disciplina é responsável por agrupar questões de uma determinada área, para que o professor seja capaz de gerar essa lista de forma aleatória que seja condizente
			com a área da disciplina.
	
		\subsection{Tópico}
			Referencia a um tópico específico de uma determinada disciplina, por exemplo, alocação dinâmica de memória é um tópico da disciplina Programação.
	
		\subsection{Questão}
			Está relacionada a uma disciplina e um tema e contém a pergunta e sua resposta.
	
		\subsection{Lista}
			Uma coleções de questões. Possui um ID e uma coleção de questões.
		
		\subsection{Diagrama de Classes}
			\begin{figure}[h]
				\centering
				\includegraphics[scale=.5]{Imagens/DiagramaDeClasses}
				\caption{Diagrama de classes}
				\label{fig:diagramaclasses}
			\end{figure}
			
		\pagebreak
		
		\subsection{Diagrama de Pacotes}
		\begin{figure}[h]
			\centering
			\includegraphics[scale=.7]{Imagens/DiagramaDePacotes}
			\caption{Diagrama de pacotes}
			\label{fig:diagramapacotes}
		\end{figure}
			

\part{Banco de Dados}
	\section{O Banco de Dados}
		O banco de dados da aplicação será o MySQL, que está incluído na solução Laragon\footnote{Disponível em \url{http://laragon.org/}} para a plataforma Windows e possui versões tanto para Mac quanto distribuições Linux. Outro motivo que levou a escolha da tecnologia foi por ser de código aberto, com amplo suporte da comunidade.
		
		\subsection{Diagrama Entidade-Relacionamento}
			\begin{figure}[h]
				\centering
				\includegraphics[scale=.6]{Imagens/Entidaderelacionamento}
				\caption{Diagrama Entidade-Relacionamento do banco de dados da aplicação}
				\label{fig:Entidaderelacionamento}
			\end{figure}
	

\part{Protótipos}
	\section{Prototipação}
		Nesta seção são apresentados os protótipos para cada interação e casos de uso apresentados anteriormente.
		
		\subsection{O Laravel}
			Como requisito do projeto, essa solução será desenvolvida usando um \textit{framework} que tem como base o padrão \textit{Model-View-Controller}, ou MVC.
			Devida a experiência dos integrantes da equipe, será usado o Laravel\footnote{Disponível em: \url{http://laravel.com/}. Acesso em 16 de novembro de 2015.}
	
		\subsection{Tela de Login}
			\begin{figure}[h]
				\centering
				\includegraphics[scale=.8]{Imagens/telaLogin}
				\caption{Tela de login}
				\label{fig:telaLogin}
			\end{figure}

		\pagebreak
	
		\subsection{Telas do Professor}
			\begin{figure}[h]
				\centering
				\includegraphics[scale=.31]{Imagens/telaProfessor}
				\caption{Tela do Professor}
				\label{fig:telaProfessor}
			\end{figure}
			
			\subsubsection{Adicionar Disciplina}
				\begin{figure}[h]
					\centering
					\includegraphics[scale=.31]{Imagens/telaDisciplina}
					\caption{Tela Adicionar Disciplina}
					\label{fig:telaDisciplina}
				\end{figure}
			
			\subsubsection{Adicionar Tópico}
				\begin{figure}[h]
					\centering
					\includegraphics[scale=.31]{Imagens/telaTopico}
					\caption{Tela de Adicionar Tópico}
					\label{fig:telaTopico}
				\end{figure}
				
				\pagebreak
				
			\subsubsection{Adicionar Questão}
				\begin{figure}[h]
					\centering
					\includegraphics[scale=.31]{Imagens/telaQuestao}
					\caption{Tela de Adicionar Questão}
					\label{fig:telaQuestao}
				\end{figure}
				
			\subsubsection{Geração de Listas}
				\begin{figure}[h]
					\centering
					\includegraphics[scale=.31]{Imagens/telaGerarLista}
					\caption{Tela de Geração de Listas}
					\label{fig:telaGerarLista}
				\end{figure}
				
			\pagebreak
				
		\subsection{Tela do Aluno}
			\begin{figure}[h]
				\centering
				\includegraphics[scale=.8]{Imagens/telaAluno}
				\caption{Tela do Aluno}
				\label{fig:telaAluno}
			\end{figure}

\part{Cronograma}

	\begin{table}[]
		\centering
		\begin{tabular}{|l|c|c|c|c|}
		\hline
		& \multicolumn{1}{l|}{\textbf{Documentação}} & \multicolumn{1}{l|}{\textbf{Pesquisa}} & \multicolumn{1}{l|}{\textbf{Implementação}} & \multicolumn{1}{l|}{\textbf{Testes}} \\ \hline
		\textit{Semana 1} 
		 & X & X& \multicolumn{1}{l|}{} &\multicolumn{1}{l|}{}                  \\ \hline
			\textit{Semana 2}  & X & X &   &   \\ \hline
			\textit{Semana 3}  & X & X &   &   \\ \hline
			\textit{Semana 4}  & X & X &   &   \\ \hline
			\textit{Semana 5}  & X & X & X &   \\ \hline
			\textit{Semana 6}  & X & X & X &   \\ \hline
			\textit{Semana 7}  & X & X & X &   \\ \hline
			\textit{Semana 8}  & X & X & X & X \\ \hline
			\textit{Semana 9}  & X & X & X &   \\ \hline
			\textit{Semana 10} & X & X & X &   \\ \hline
			\textit{Semana 11} & X & X & X &   \\ \hline
			\textit{Semana 12} & X & X & X & X \\ \hline
			\textit{Semana 13} & X & X & X &   \\ \hline
			\textit{Semana 14} & X & X & X & X \\ \hline
		\end{tabular}
	\caption{Cronograma}
	\label{cronograma}
	
\end{table}

\part*{Anexos}
\appendix
	\section{Login}
		\begin{mdframed}[linecolor=black, topline=true, bottomline=true,leftline=false, rightline=false, backgroundcolor=white,userdefinedwidth=\textwidth]
			\inputminted[linenos=true,numberblanklines=true,showspaces=false,breaklines=true]{html}{Codigos/login.html}
		\end{mdframed}
		
		\pagebreak
		
	\section{Telas do Professor}
		\begin{mdframed}[linecolor=black, topline=true, bottomline=true,leftline=false, rightline=false, backgroundcolor=white,userdefinedwidth=\textwidth]
			\inputminted[linenos=true,numberblanklines=true,showspaces=false,breaklines=true]{html}{Codigos/professor.html}
		\end{mdframed}
		
		\pagebreak
		
	\section{Adicionar Disciplina}
		\begin{mdframed}[linecolor=black, topline=true, bottomline=true,leftline=false, rightline=false, backgroundcolor=white,userdefinedwidth=\textwidth]
			\inputminted[linenos=true,numberblanklines=true,showspaces=false,breaklines=true]{html}{Codigos/disciplina.html}
		\end{mdframed}
		
		\pagebreak
		
	\section{Adicionar Tópico}
		\begin{mdframed}[linecolor=black, topline=true, bottomline=true,leftline=false, rightline=false, backgroundcolor=white,userdefinedwidth=\textwidth]
			\inputminted[linenos=true,numberblanklines=true,showspaces=false,breaklines=true]{html}{Codigos/topico.html}
		\end{mdframed}
		
		\pagebreak
		
	\section{Adicionar Questão}
		\begin{mdframed}[linecolor=black, topline=true, bottomline=true,leftline=false, rightline=false, backgroundcolor=white,userdefinedwidth=\textwidth]
			\inputminted[linenos=true,numberblanklines=true,showspaces=false,breaklines=true]{html}{Codigos/questao.html}
		\end{mdframed}
		
		\pagebreak
		
	\section{Geração de Listas}
		\begin{mdframed}[linecolor=black, topline=true, bottomline=true,leftline=false, rightline=false, backgroundcolor=white,userdefinedwidth=\textwidth]
			\inputminted[linenos=true,numberblanklines=true,showspaces=false,breaklines=true]{html}{Codigos/gerar.html}
		\end{mdframed}
		
		\pagebreak
			
	\section{Tela do Aluno}
		\begin{mdframed}[linecolor=black, topline=true, bottomline=true,leftline=false, rightline=false, backgroundcolor=white,userdefinedwidth=\textwidth]
			\inputminted[linenos=true,numberblanklines=true,showspaces=false,breaklines=true]{html}{Codigos/aluno.html}
		\end{mdframed}
		
% Referências

\end{document}          
